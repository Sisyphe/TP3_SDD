\documentclass[12pt]{report}
%\documentclass[twoside, openright, 12pt]{report}
 
\usepackage[top=2cm, bottom=2cm, left=2cm, right=2cm]{geometry} % definition des marges

\usepackage[utf8]{inputenc}
\usepackage[T1]{fontenc}
\usepackage[francais]{babel} 

\usepackage{url}
\usepackage{graphicx}
\usepackage{titlepic}
\usepackage{titlesec, blindtext, color}
\usepackage{amsmath}
\usepackage{amssymb}
\usepackage{mathrsfs}
\usepackage{listings}
\usepackage{multirow}
\usepackage{caption}
\usepackage{minitoc}
\usepackage{perpage}
\usepackage{setspace}
\setstretch{1.5} % interligne de 1.5

%paragraph options
\setlength{\parindent}{15pt}
\setlength{\parskip}{6pt}

% enleve la mention chapitre
\titleformat{\chapter}[hang]{\bf\huge}{\thechapter}{2pc}{}

% Ligne orpheline ou veuve
\widowpenalty=10000 % empeche au maximum la coupure avant la derniere ligne 
\clubpenalty=10000 % empeche au maximum la coupure apres la premiere ligne 
\raggedbottom

\lstset{language=C, basicstyle=\scriptsize, numbers=left, numberstyle=\scriptsize, numbersep=12pt, caption=\lstname, captionpos=b}

\def\maketitle{
\thispagestyle{empty}

\noindent 
  Gérald Lelong \\
  Noël Martignoni \\
  G31 \\
 
  \vskip 6cm
  
  \begin{center}\leavevmode
  \Large  \textbf{Structure de donnée} \\
  \rule{9cm}{1pt} \\
  \Huge Compte rendu TP3 \\
  \end{center}
      
  \cleardoublepage
  }  

% Le document --------------------------------------------------------------------------------------------------------
\begin{document}

\maketitle

\tableofcontents

\chapter{Présentation du TP}

\section{Description}

L'objectif de ce TP est de créer une structure de donnée permettant de stocker une matrice creuse lue depuis un fichier texte dans une table.

\chapter{Programme C}

\section{matrix.h}

\lstinputlisting[language=C]{../matrix.h}

\section{matrix.c}

\lstinputlisting[language=C]{../matrix.c}

\chapter{Compilation et tests}

\section{Makefile}

\lstinputlisting[language=make]{../Makefile}

\section{Jeux de test}

\subsection{Ajout d'une valeur dans une matrice vide}

\lstinputlisting[language=make]{../empty_matrix.dat}

\lstinputlisting[language=make]{../tests/insertion_matrice_vide.c}

\begin{lstlisting}[caption=Sortie]
********** Matrix **********

row : 23
    col : 44 value : 90

*****************************
\end{lstlisting}

La valeur est bien ajoutée à la matrice.

\subsection{Lecture d'un fichier vide}

\lstinputlisting[language=make]{../tests/lecture_fichier_vide.c}

Aucune erreur de segmentation lors de l'exécution.

\subsection{Lecture d'un fichier dans le désordre}

\lstinputlisting[language=make]{../matrix.dat}

\lstinputlisting[language=make]{../tests/lecture_desordre.c}

\begin{lstlisting}[caption=Sortie]
    ********** Matrix **********

row : 5
    col : 8 value : -4
row : 12
    col : 2 value : 256
    col : 14 value : -51
    col : 23 value : 111
row : 25
    col : 5 value : 311
    col : 18 value : 79
row : 55
    col : 21 value : -123
    col : 31 value : -45

*****************************
\end{lstlisting}

La matrice est conforme au fichier d'entrée.

\end{document}
